\documentclass[twocolumn]{article}

\usepackage[margin = 1in, paperwidth = 8.5in, paperheight = 11in]{geometry}
\usepackage{amsmath}
\usepackage{graphicx}
\usepackage{xcolor}
\usepackage[utf8]{inputenc}
\usepackage{hyperref}
\usepackage{mathpazo}
\newcommand{\todo}[1]{\textcolor{red}{#1}}


\title{CS482/682 Final Project Proposal\\
\large Using Unsupervised Learning in Classification of Autonomous Vehicles}

\author{Tianyu Ding, Ivan Liao, Dylan Madisetti, Alex Sun}

\date{}

\begin{document}

  \maketitle

  \section{Problem Statement}

  It is well known that in autonomous driving the position of nearby
automobiles is a key question for autonomous vehicles.  As a case study in how
unsupervised learning can be used to tackle established, benchmarkable
problems, this project seeks to compete in the
\href{https://www.kaggle.com/c/pku-autonomous-driving}{Peking University/Baidu
Autonomous Vehicle Challenge}.. We are interested in using advanced
unsupervised learning techniques to help prediction of the model type and
position of the captured cars.

  \vspace{-.1in} \section{Dataset}

  The above challenge provides dataset containing photos of streets, taken from
the roof of a car. In particular, we have \begin{itemize} \item the images of
cars and their \emph{pose} information \item  \emph{segmentation masks} of the
cars against the background \item \emph{3D models} of the unmasked cars of
interest \end{itemize}


  \section{Outline of Solving the Problem}

   An unsupervised approach to solving this problem is proposed by a 3 stage
progressive pipeline. Each stage is a standalone, unsupervised methodology and
thus allows for incremental progress.

  \textbf{Stage 1}. In the first stage, images of traffic will be segmented
into distinct cars and the background. To achieve this, a U-net like model will
provide segmentation of all the cars with respect to the background. Then, an
unsupervised model (K-Mean or Gaussian Mixture) will be applied to the
segmented mask in obtaining the distinct clusters of cars \cite{Ji2019}.

  \textbf{Stage 2}. In the second stage, an embedding space of masked cars will
be produced. Using the mask, individual cars will be extracted (bounding box)
from the background to form a dataset. A convolutional neural network, followed
by dense layers will map this segmentation into some latent space, which
be "grown" into a representation as with standard auto-encoders.

  In the context of autonomous driving, the latent space will be clustered to
provide transparency into the structure of the data relations based on the
semantic distribution of the dataset such as the type (SUV or Sedan), the model
or the shape of the car. Note that this stage can occur independently of Stage 1
using the provided label mask. Completion of this task should be a fully
functional auto-encoder, and a beautiful (hopefully) map of the latent space
using a method like in \cite{Maaten2008}.

  \textbf{Stage 3} The final aspect of the pipeline produces a labeled
embedding space. The segmentation vectors in the space produced by Stage 2,
will be mapped into a new space shared with learned model labels, such as the
position, yaw, pitch or the roll. This idea is borrowed from
\cite{Henderson2017} and can be bested expressed as semi-supervised learning.
This methodology is chosen such that an unspecified number of models can be
used in classification. In a given batch (with unique labels), a softmax loss
can be used by matching the labels to the given sample. Further work includes
developing means of visualizing the latent "car space" by implementing
information from the provided 3D models.

\begin{thebibliography}{8} \bibitem{Maaten2008} Van der Maaten, L. and Hinton,
G. Visualizing Data using t-SNE. \textit{Journal of Machine Learning Research},
(9) 2579-2605, 2008.

  \bibitem{Ji2019} Ji, X. and Henriques, J. and Vedaldi, A. Invariant
Information Clustering for Unsupervised Image Classification and Segmentation.
\textit{The IEEE International Conference on Computer Vision (ICCV)},2019.

  \bibitem{Henderson2017} Henderson, M. and Al-Rfou, R. and Strope, B. and
Sung, Y. and Lukacs, L. and Guo, R. and Kumar, S. and Miklos, B. and Kurzweil,
R.  Efficient Natural Language Response Suggestion for Smart Reply.
\textit{arXiv:1705.00652}.

\end{thebibliography}
\end{document}
